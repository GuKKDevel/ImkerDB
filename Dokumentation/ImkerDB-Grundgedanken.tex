\section{Grundlegende Gedanken}

Um die Vorgänge in einer (Hobby-)Imkerei vernünftig modellieren zu können, sind einige Vorüberlegungen über die anfallenden Arbeiten notwendig. Zum einen ist da die Tierhaltung, auch Völkerführung oder Betriebsweise genannt, zum anderen gibt es noch --neben der Honiggewinnung-- die Königinnenzucht, den Verkauf von Völkern (Ableger/Wirtschaftsvölker), die Propolisgewinnung und die Wachsgewinnung. \bigskip 

Für einige dieser Tätigkeiten unterliegt der Imker bestimmten gesetzlichen Vorschriften und ist daher verpflichtet, in diesen Bereichen seine Handlungen zu dokumentieren.\medskip

Dazu gibt es zum einen das Bestandsbuch, in dem der Imker sämtliche Bahandlungen seiner Bienenvölker mit apothekenpflichtigen Medikamente eintragen muß. \medskip

Zum anderen ist er verpflichtet, ein sogenanntes Honigbuch zu führen, aus dem jederzeit ersichtlich ist, was er mit dem Honig gemacht hat, von der Ernte bis zur Abgabe des Gebindes (Glas o.ä). Da der (Hobby-)Imker ein Lebensmittel in Verkehr bringt, wenn er seinen Honig verkauft oder selbst wenn er ihn nur verschenkt. \bigskip
 
In dieser Version des Dokuments geht es primär um die Honigernte, speziell durch Schleuderung.


\section{Honigernte}
\paragraph{Ernte von Scheibenhonig}
Bei dieser Art der Honigernte werden nur Waben im Naturbau verwendet. Diese Waben, die in nicht gedrahteten Rähmchen ausgebaut wurden, werden aus dem Rähmchen gelöst und dann portionsweise abgeschnitten und direkt in Schälchen oder sonstige Verkaufsgebinde abgefüllt.
\paragraph{Honigernte durch Pressen}
Die Waben werden aus dem Bienenstock genommen, aus den Rähmchen geschnitten und in einer Presse ausgepresst, ohne dass sie vorher entdeckelt wurden. Der ausgepresste Honig wird in einem Sammelgefäß aufgefangen.
\paragraph{Honigernte durch Tropfen lassen}
Die Waben werden aus dem Volk genommen, entdeckelt und zum Austropfen über ein Sammellgefäß gehängt. 
\paragraph{Honigernte durch Schleuderung}
Die Waben werden aus dem Volk entnommen, entdeckelt und in einer Schleuder ausgeschleudert und in einem Gefäß gesammelt.




 